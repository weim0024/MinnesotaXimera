\documentclass[number]{ximera}

%My github password is ximera1
%My GeoGebra password is Ximera1

%\usepackage{tikz}

\renewcommand{\theenumi}{\arabic{enumi}.}


% set font encoding for PDFLaTeX, XeLaTeX, or LuaTeX
\usepackage{ifxetex,ifluatex}
\newif\ifxetexorluatex
\ifxetex
  \xetexorluatextrue
\else
  \ifluatex
    \xetexorluatextrue
  \else
    \xetexorluatexfalse
  \fi
\fi

\ifxetexorluatex
  \usepackage{fontspec}
\else
  \usepackage[T1]{fontenc}
  \usepackage[utf8]{inputenc}
  \usepackage{lmodern}
\fi

\usepackage{hyperref}
\usepackage{tikz} %% Adding this here so it's not manually added elsewhere

% set font encoding for PDFLaTeX, XeLaTeX, or LuaTeX
%\usepackage{ifxetex,ifluatex}
%\newif\ifxetexorluatex
%\ifxetex
%  \xetexorluatextrue
%\else
%  \ifluatex
%    \xetexorluatextrue
%  \else
%    \xetexorluatexfalse
%  \fi
%\fi
%
%\ifxetexorluatex
%  \usepackage{fontspec}
%\else
%  \usepackage[T1]{fontenc}
%  \usepackage[utf8]{inputenc}
%  \usepackage{lmodern}
%\fi
%
%\usepackage{hyperref}

\usepackage{tikz}

\title{Vector Form of Area}
\author{Univ. of Minnesota MathCEP}


% Enable SageTeX to run SageMath code right inside this LaTeX file.
% http://mirrors.ctan.org/macros/latex/contrib/sagetex/sagetexpackage.pdf
% \usepackage{sagetex}

\begin{document}

\begin{abstract}
  In this activity, you will discover a simple formula for the area when two of the sides of the triangle are given as vectors that begin at the origin.
\end{abstract}

\maketitle

Suppose we have a triangle with one vertex at the origin. We can express the other two sides of the triangle as vectors.

We wish to write the area of the triangle in terms of $a,b,c,d$

\bigskip

\begin{tikzpicture}[scale=1]
\draw [->] (0,0) -- node[above left]{$\overrightarrow u$} (4,6);
\draw [->] (0,0) -- node[below]{$\overrightarrow v$} (5,1);
 \draw [dashed] (5,1) -- (4,6);
\node at (-0.5,0) {$(0,0)$};
\node at (5.8,1) {$(c,d)$};
\node at (4.8,6) {$(a,b)$};
\node at (0.7,0.5) {$\theta$};
 \end{tikzpicture}

\begin{problem}
Write the area of the triangle using $\sin \theta$, $||\overrightarrow u||$ and $||\overrightarrow v||$.

Area = $\answer{\frac{||\vec{u}||\cdot||\vec{v}||\sin(\theta)}{2}}$
\begin{hint}
Use the formula for the area of an SAS triangle, where the side lengths are $||\overrightarrow u||$ and $||\overrightarrow v||$, and $\theta$ is the angle between them.
\end{hint}
\end{problem}

\begin{problem}
Find the direction of $\overrightarrow u$. That is, find $\theta_u$.

$\theta_u=\answer{\tan^{-1}\left(\frac{b}{a}\right)}$
\begin{hint}
Your answer should be in terms of the variables $a$ and $b$.
\end{hint}
\begin{hint}
Which trig function relates an angle to the side-lengths of the triangle that it's a part of?
\end{hint}
\end{problem}

\begin{problem}
Find the direction of $\overrightarrow v$. That is, find $\theta_v$.

$\theta_v=\answer{\tan^{-1}\left(\frac{d}{c}\right)}$
\begin{hint}
You answer should be in terms of the variables $c$ and $d$.
\end{hint}
\begin{hint}
Which trig function relates an angle to the side-lengths of the triangle that it's a part of?
\end{hint}
\end{problem}

\begin{problem}
Express $\theta$ in terms of $\theta_u$ and $\theta_v$.

$\theta = \answer{\theta_u-\theta_v}$ %%%This answer can be ambiguous. Which one should I input? Or maybe make them all freeResponse?
\begin{hint}
What angle would you need to subtract from $\theta_u$ to get to $\theta_v$?
\end{hint}
\end{problem}

\begin{problem}
Express $\sin \theta$ using inverse trig functions and $a,b,c,d$.

$\sin\theta=\answer{\sin(\theta_u)\cos(\theta_v)-\cos(\theta_u)\sin(\theta_v)}$
\begin{hint}
Apply the angle-difference formula for sine.
\end{hint}
\end{problem}

\begin{problem}
Use your formula from above to write $\sin \theta$ in terms of $a,b,c,d$.

$\sin\theta = \answer{\frac{bc-ad}{(a^2+b^2)(c^2+d^2)}}$
\begin{hint}
At first, you may have some compositions of trig functions with inverse trig functions. Draw the triangles determined by these expressions first to simplify your answer.
\end{hint}
\end{problem}

\begin{problem}
Write the area of the triangle in terms of $a,b,c,d$.

Area = $\answer{\frac{bc-ad}{2}}$
\begin{hint}
Refer back to your formula for area in Problem 1.
\end{hint}
\begin{hint}
Rewrite $||\vec{u}||$ and $||\vec{v}||$ in terms of the variables $a,b,c,d$.
\end{hint}
\end{problem}

\begin{exercise}
If $\overrightarrow u = (5,1)$ and $\overrightarrow v = (4,-2)$, find the area. 

Area = $\answer{7}$
\end{exercise}

\begin{exercise}
If $\overrightarrow u = (4,-2)$ and $\overrightarrow v = (5,1)$, find the area.  

Area = $\answer{-7}$
\end{exercise}

\begin{question}
Compare your answers to Exercises 8 and 9. Why are they different? Do you need to adjust your formula?
\begin{freeResponse}

\end{freeResponse}
\end{question}

\end{document}

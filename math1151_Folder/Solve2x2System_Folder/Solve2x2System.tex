\documentclass[number]{ximera}

%My github password is ximera1
%My GeoGebra password is Ximera1

% set font encoding for PDFLaTeX, XeLaTeX, or LuaTeX
\usepackage{ifxetex,ifluatex}
\newif\ifxetexorluatex
\ifxetex
  \xetexorluatextrue
\else
  \ifluatex
    \xetexorluatextrue
  \else
    \xetexorluatexfalse
  \fi
\fi

\ifxetexorluatex
  \usepackage{fontspec}
\else
  \usepackage[T1]{fontenc}
  \usepackage[utf8]{inputenc}
  \usepackage{lmodern}
\fi

\usepackage{hyperref}

\title{Solving Systems of Linear Equations}
\author{MathCEP}

% Enable SageTeX to run SageMath code right inside this LaTeX file.
% http://mirrors.ctan.org/macros/latex/contrib/sagetex/sagetexpackage.pdf
% \usepackage{sagetex}




\begin{document}

\begin{abstract}
  In this activity, you will discover the issue involved in using the Law of Sines and Law of Cosines to solve triangles.
\end{abstract}

\maketitle

\begin{enumerate}

\item Person 1 - Solve the following system of linear equations for $x$.

\begin{center}
$ax+by=p$

$cx+dy=q$
\end{center}

by doing the following.

\bigskip

\begin{itemize}
\item Multiply the top equation by $d$.
\item Multiply the bottom equation by $-b$
\item Add the two new equations.
\item Solve for $x$
\end{itemize}

Person 2 - Solve the following system of linear equations for $y$.

\begin{center}
$ax+by=p$

$cx+dy=q$
\end{center}

by doing the following.

\bigskip

\begin{itemize}
\item Multiply the top equation by $-c$.
\item Multiply the bottom equation by $a$
\item Add the two new equations.
\item Solve for $y$
\end{itemize}

\item Solve the following system of linear equations for $x$.

\begin{center}
$ex+fy+gz=r$

$hx+iy+jz=s$

$kx+ly+mz=t$
\end{center}

by doing the following.

\bigskip

\begin{itemize}
\item Multiply the top equation by $-j$.
\item Multiply the middle equation by $g$
\item Add the two new equations to get equation (*).
\item Multiply the bottom equation by $-j$.
\item Multiply the middle equation by $m$
\item Add the two new equations to get equation (**).
\item Solve the system of equations (*) and (**).
\end{itemize}
\end{enumerate}
\end{document}
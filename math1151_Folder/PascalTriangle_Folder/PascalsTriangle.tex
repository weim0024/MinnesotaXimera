\documentclass[number]{ximera}

%My github password is ximera1
%My GeoGebra password is Ximera1

% set font encoding for PDFLaTeX, XeLaTeX, or LuaTeX
\usepackage{ifxetex,ifluatex}
\newif\ifxetexorluatex
\ifxetex
  \xetexorluatextrue
\else
  \ifluatex
    \xetexorluatextrue
  \else
    \xetexorluatexfalse
  \fi
\fi

\ifxetexorluatex
  \usepackage{fontspec}
\else
  \usepackage[T1]{fontenc}
  \usepackage[utf8]{inputenc}
  \usepackage{lmodern}
\fi

\usepackage{hyperref}

\title{Pascal's Triangles}
\author{Univ. of Minnesota MathCEP}


% Enable SageTeX to run SageMath code right inside this LaTeX file.
% http://mirrors.ctan.org/macros/latex/contrib/sagetex/sagetexpackage.pdf
% \usepackage{sagetex}

\begin{document}

\begin{abstract}
In this activity, you will make connections between arranging colored balls and entries in Pascal's Triangle.
By the end of this activity, you should understand the role that recursion plays in Pascal's Triangle (e.g., how each entry depends on entries before it).
\end{abstract}

\maketitle

\begin{question}
In how many ways can two colored balls, red (R) and blue (B) be arranged in order?

Answer: $\answer{2}$ ways
\begin{hint}
$RB$ and $BR$
\end{hint}
\end{question}

\begin{question}
In how many ways can three colored balls, red (R), blue (B) and yellow (Y) be arranged in order? 

Answer: $\answer{6}$ ways

Write out the complete list:
\begin{freeResponse}
\end{freeResponse}
\begin{hint}
Try starting by writing all of the arrangements that start with red (R).
\end{hint}
\end{question}

\begin{question}
In how many ways can four colored balls, red (R), blue (B), yellow (Y) and green (G) be arranged in order?
\begin{itemize}
\item Person 1 - Write out the possibilities that begin with R. How many are there? $\answer{6}$
\item Person 2 - Write out the possibilities that begin with B. How many are there? $\answer{6}$
\item Person 3 - Write out the possibilities that begin with Y. How many are there? $\answer{6}$
\item Person 4 - Write out the possibilities that begin with G. How many are there? $\answer{6}$
\end{itemize}
Hence, there are $\answer{24}$ ways that four colored balls can be arranged in order.
\end{question}

\begin{question}
In how many ways can five colored balls, red (R), blue (B), yellow (Y), green (G) and purple (P) be arranged in order?
\begin{itemize}
\item Person 1 - Write out the possibilities that begin with PR.
\item Person 2 - Write out the possibilities that begin with PB.
\item Person 3 - Write out the possibilities that begin with PY.
\item Person 4 - Write out the possibilities that begin with PG.
\end{itemize}
There are $\answer{120}$ ways five colored balls can be arranged in order.
\begin{hint}
How do your answers to this question compare to your answers for the previous question?
\end{hint}
\begin{hint}
As you write out the possibilities that begin with PR, PB, etc., find a pattern that helps you keep track of which ones you've already written down.
\end{hint}
\end{question}

\begin{question}
In how many ways can $n$ colored balls be arranged in order? There are $\answer{n!}$ ways.
\begin{hint}
Your answer should be in terms of $n$.
\end{hint}
\end{question}

\begin{question}
In how many ways can three balls from among five colored balls, red (R), blue (B), yellow (Y), green (G) and purple (P) be arranged in order?
\begin{itemize}
\item Person 1 - Write out the possibilities that begin with R.
\item Person 2 - Write out the possibilities that begin with B.
\item Person 3 - Write out the possibilities that begin with Y.
\item Person 4 - Write out the possibilities that begin with G.
\item Person 5 - Write out the possibilities that begin with P.
\end{itemize}
There are $\answer{60}$ ways.
\begin{hint}
One of the possibilities that begins with R is RBY.
\end{hint}
\end{question}

\begin{question}
In how many ways can $k$ balls from among $n$ colored balls be arranged in order?

There are $\answer{\frac{n!}{(n-k)!}}$ ways.
\begin{hint}
There are $n!$ ways to arrange $n$ objects in order. If we are only looking to arrange $k$ of those balls in order, how many arrangements are unnecessary? %%% I don't know if I like this hint
\end{hint}
\end{question}

\begin{question}
How many of the permutations in Problem 6 contain the colors R,G,B in some order? 

In other words, there are $\answer{6}$ ways to order the colors R,G,B. 

How is this related to Problem 2? 
\begin{freeResponse}
\end{freeResponse}
\end{question}

\begin{question}
In how many ways can $k$ balls from among $n$ colored balls be chosen, if order doesn't matter? 
In other words, how many combinations are there?

There are $\answer{\frac{n!}{(n-k)!k!}}$ ways.
\begin{hint}
How many arrangements of letter (in order) are treated as the same? Your expression will need to eliminate these redundant combinations.
\end{hint}
\end{question}

The notation for the number of combinations of $k$ balls from a total of $n$ balls is read as '$n$ choose $k$' and is denoted by

\[
\left( {\begin{array}{*{20}c}
n \\
k\\
\end{array}} \right).
\]

\begin{exercise}
${6\choose{3}} = \answer{20}$
\end{exercise}

\begin{exercise}
${9 \choose{2}}=\answer{36}$
\end{exercise}



\begin{problem}
Show the recursion in Pascal's Triangle works for combinations in this example: Show that the number of combinations of 4 colors chosen from 10 equals the number of combinations of 4 colors chosen from 9 plus the number of combinations of 3 colors chosen from 9. 
\begin{hint}Examine the 10th color. In how many combinations is the 10th color present?
\end{hint}
\begin{hint}
In how many combinations is the 10th color not present?
\end{hint}
\begin{freeResponse}
\end{freeResponse}
\end{problem}
\end{document}









\end{document}

\documentclass[number]{ximera}

%My github password is ximera1
%My GeoGebra password is Ximera1

% set font encoding for PDFLaTeX, XeLaTeX, or LuaTeX
\usepackage{ifxetex,ifluatex}
\newif\ifxetexorluatex
\ifxetex
  \xetexorluatextrue
\else
  \ifluatex
    \xetexorluatextrue
  \else
    \xetexorluatexfalse
  \fi
\fi

\ifxetexorluatex
  \usepackage{fontspec}
\else
  \usepackage[T1]{fontenc}
  \usepackage[utf8]{inputenc}
  \usepackage{lmodern}
\fi

\usepackage{hyperref}

\title{Angle Sum Formulas - Part II}
\author{Univ. of Minnesota MathCEP}


% Enable SageTeX to run SageMath code right inside this LaTeX file.
% http://mirrors.ctan.org/macros/latex/contrib/sagetex/sagetexpackage.pdf
% \usepackage{sagetex}

\begin{document}

\begin{abstract}
  In this activity, you will discover the issue involved in using the Law of Sines and Law of Cosines to solve triangles.
\end{abstract}

\maketitle

\begin{enumerate}

\item In Part I, we developed the following formulas:

\[ \cos(A+B) = \cos A \cos B - \sin A \sin B \]

\[ \cos(A-B) = \cos A \cos B + \sin A \sin B \]

Use the first formula to find a formula for

\[ \cos(90^\circ + \theta) \]

\item

It is also true that 

\[ \sin(90^\circ + \theta) = \cos \theta \]

(This can be a bonus problem to be proved by the students)

\item Let $\theta = M+N$. 

Use the formula for $\cos(90^\circ + \theta)$ to show that

\[\sin(M+N) = - \cos(90^\circ + M + N) \]

\item Use the formula for $\cos(A+B)$, letting $A = 90^\circ + M$ and $B = N$ to write a formula for $ - \cos(90^\circ + M + N) $

\item Simplify the previous expression using the formulas found in problems 1 and 2.

\item Combine problems 3, 4 and 5 to derive a formula for $\sin(M+N)$.

\item In problem 3) of Part I, you used a process to convert the formula for $\cos(A-B)$ to a formula for $\cos(A+B)$. Use this same process to convert the formula for $\sin(M+N)$ to a formula for $\sin(M-N)$.

\newpage

\item Summarize your results:

\begin{itemize}

\item $\cos(A-B) = $

\vskip 0.5 cm

\item $\cos(A+B) = $

\vskip 0.5 cm

\item $\sin(A-B) = $

\vskip 0.5 cm

\item $\sin(A+B) = $

\end{itemize}

\item Find formulas for $\tan (A-B)$ and $\tan (A+B)$ as follows:

\begin{enumerate}

\item Write $\tan (A-B)$ as $\quad \displaystyle{\frac{\sin (A-B)}{\cos (A-B)}}$, then use 8) to rewrite in terms of $\sin A$, $\sin B$, $\cos A$ and $\cos B$.

\item Divide both the numerator and denominator by $\cos (A) \cos(B)$ and simplify.

\item Rewrite in terms of $\tan (A)$ and $\tan(B)$. 

\item Repeat the process for $\tan (A+B)$

\item Summarize your results.

\vskip 0.25 cm

\begin{itemize}

\item $\tan(A-B) = $

\vskip 0.5 cm

\item $\tan(A+B) = $

\end{itemize}


\end{enumerate}

\end{enumerate}
\end{document}

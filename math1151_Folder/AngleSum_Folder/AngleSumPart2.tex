\documentclass[number]{ximera}

%My github password is ximera1
%My GeoGebra password is Ximera1

% set font encoding for PDFLaTeX, XeLaTeX, or LuaTeX

%\usepackage{tikz}

\renewcommand{\theenumi}{\arabic{enumi}.}


% set font encoding for PDFLaTeX, XeLaTeX, or LuaTeX
\usepackage{ifxetex,ifluatex}
\newif\ifxetexorluatex
\ifxetex
  \xetexorluatextrue
\else
  \ifluatex
    \xetexorluatextrue
  \else
    \xetexorluatexfalse
  \fi
\fi

\ifxetexorluatex
  \usepackage{fontspec}
\else
  \usepackage[T1]{fontenc}
  \usepackage[utf8]{inputenc}
  \usepackage{lmodern}
\fi

\usepackage{hyperref}


\title{Angle Sum Formulas - Part II}
\author{Univ. of Minnesota MathCEP}


% Enable SageTeX to run SageMath code right inside this LaTeX file.
% http://mirrors.ctan.org/macros/latex/contrib/sagetex/sagetexpackage.pdf
% \usepackage{sagetex}

\begin{document}

\begin{abstract}
In the previous activity, we derived formulas for $\cos(A+B)$ and $\cos(A-B)$ in terms of $\cos(A),\sin(A),\cos(B)$, and $\sin(B)$.
In this activity, we will use those formulas to derive angle-sum and angle-difference formulas for the sine and tangent functions.
\end{abstract}

\maketitle

The goal of the first half of this activity is to derive the angle-sum and angle-difference formulas for sine.
Recall that in the activity ``Angle Sum Formulas - Part I,'' we developed formulas for the cosine of the sum and difference of two arbitrary angles:
\[ \cos(A+B) = \cos A \cos B - \sin A \sin B \]

\[ \cos(A-B) = \cos A \cos B + \sin A \sin B \]

\begin{problem}
Use the formula for $\cos(A+B)$ to find an expression for $\cos(90^\circ + \theta)$.
\[\cos(90^\circ + \theta) = \answer{-\sin(\theta)}\]
\begin{hint}
Rewrite the expression with $A=90^\circ$ and $B=\theta$. Are there any expressions we can evaluate in order to simplify the formula?
\end{hint}
\begin{hint}
What is $\cos(90^\circ)$? What is $\sin(90^\circ)$?
\end{hint}

\end{problem}

It is also true that \[\sin(90^\circ+\theta) = \cos(\theta).\]
\begin{question}
(Optional) Why does $\sin(90^\circ+\theta)=\cos(\theta)$?
\begin{freeResponse}

\end{freeResponse}
\end{question}

You will use this information to develop formulas for the sine of the sum and difference of two angles.

\begin{problem}
Let $\theta = M+N$. Use the formula for $\cos(90^\circ+\theta)$ to show that
\[\sin(M+N)=-\cos(90^\circ+M+N).\]
By the expression you derived in Problem 1 for $\cos(90^\circ+\theta)$,
\[\cos(90^\circ+M+N) = \answer{-\sin(M+N)}.\]
Hence,
\[\sin(M+N)=\answer{-\cos(90^\circ+M+N)}.\]
\begin{hint}
In the expression for $\cos(90^\circ+\theta)$, replace instances of $\theta$ with $M+N$.
\end{hint}
\end{problem}

\begin{problem}
Start with the formula for $\cos(A+B)$, letting $A=90^\circ+M$ and $B=N$ to write a formula for $-\cos(90^\circ+M+N)$.
\[\cos(90^\circ+M+N)=\cos(\answer{90^\circ+M})\cos(\answer{N})-\sin(\answer{90^\circ+M})\sin(\answer{N}).\]
Hence,
\[-\cos(90^\circ+M+N)=\answer{\sin(90^\circ+M)\sin(N)-\cos(90^\circ+M)\cos(N)}.\]
\begin{hint}
How does the formula for $-\cos(90^\circ+M+N)$ differ from the formula for $\cos(90^\circ+M+N)$?
\end{hint}

\end{problem}

\begin{problem}
To simplify this expression, recall the substitutions described earlier in this activity:
\begin{itemize}
\item $\cos(90^\circ+\theta)=\answer{-\sin(\theta)}$
\item $\sin(90^\circ+\theta)=\answer{\cos(\theta)}$
\end{itemize}

\end{problem}

\begin{problem}
Combine the previous two problems to simplify the expression in Problem 4:
\[-\cos(90^\circ+M+N)=\answer{\cos(M)\sin(N)+\sin(M)\cos(N)}.\]
\begin{hint}
Substitute $\theta$ for $M$.
\end{hint}

\end{problem}

\begin{problem}
Combine Problems 3-6 to derive a formula for $\sin(M+N)$.
\[\sin(M+N)=\answer{\cos(M)\sin(N)+\sin(M)\cos(N)}\]
\end{problem}

\begin{problem}
In the activity ``Angle Sum Formulas - Part I,'' you used a process to convert the formula for $\cos(A-B)$ to a formula for $\cos(A+B)$. Use this same process to convert the formula for $\sin(M+N)$ to a formula for $\sin(M-N)$.
\[\sin(M-N)=\answer{\sin(M)\cos(N)-\cos(M)\sin(N)}\]
\begin{hint}
$M-N=M+(-N)$
\end{hint}

\end{problem}

\begin{problem}
Summarize your results:
\begin{itemize}

\item $\cos(A-B) = \answer{\cos(A)\cos(B)-\sin(A)\sin(B)}$
\item $\cos(A+B) = \answer{\cos(A)\cos(B)+\sin(A)\sin(B)}$
\item $\sin(A-B) = \answer{\sin(A)\cos(B)-\cos(A)\sin(B)}$
\item $\sin(A+B) = \answer{\sin(A)\cos(B)+\cos(A)\sin(B)}$
\end{itemize}
\begin{hint}
Substitute $M=A$ and $N=B$ from the previous problem to get formulas for $\sin(A-B)$ and $\sin(A+B)$.
\end{hint}

\end{problem}


In the next part of this activity, we will use the angle sum and angle difference formulas for sine and cosine to derive the angle sum and difference formulas for tangent.

\begin{problem}
Recall the definition of $\tan(\theta)=\frac{\answer{\sin(\theta)}}{\answer{\cos(\theta)}}$.
\end{problem}

\begin{problem}
Find formulas for $\tan (A-B)$ and $\tan (A+B)$ as follows:

\begin{enumerate}
\item Write $\tan (A-B)$ as $\frac{\sin (A-B)}{\cos (A-B)}$, and rewrite that in terms of $\sin A$, $\sin B$, $\cos A$ and $\cos B$.
\item Divide both the numerator and denominator by $\cos (A) \cos(B)$ and simplify.
\item Rewrite in terms of $\tan (A)$ and $\tan(B)$. 
\item Repeat the process for $\tan (A+B)$
\item Summarize your results.
  \begin{itemize}
  \item $\tan(A-B)=\answer{\frac{\tan(A)-\tan(B)}{1+\tan(A)\tan(B)}}$
  \item $\tan(A+B)=\answer{\frac{\tan(A)+\tan(B)}{1-\tan(A)\tan(B)}}$
  \end{itemize}
\end{enumerate}

\end{problem}


\end{document}

\documentclass[numbers]{ximera}

%My github password is ximera1
%My GeoGebra password is Ximera1


%\usepackage[fleqn]{amsmath}

\title{Double Angle Formulas}
\author{Mike Weimerskirch}
\begin{document}
\begin{abstract}
  In this activity, you will discover formulas for $\sin (2\theta)$, $\cos (2\theta)$, $\sin \left(\frac{\theta}{2}\right)$ and $\cos \left(\frac{\theta}{2}\right)$, using your knowledge of the Angle Sum Formulas.
\end{abstract}

\maketitle


{\bf {Alternate Versions of the Pythagorean Theorem}}

\bigskip

$\sin^2 \theta + \cos^2 \theta = 1$

\bigskip

Subtract $\cos^2 \theta$ from both sides to get a formula for $\sin^2 \theta$ 
\begin{question}
  $\sin^2 \theta =$ 
  $\answer{1-\cos^2 \theta} $
\end{question}

Subtract $\sin^2 \theta$ from both sides to get a formula for $\cos^2 \theta$ 
\begin{question}
  $\cos^2\theta = $
  $\answer{1-\sin^2\theta}$
\end{question}



\vfill

{\bf {Double Angle Formula for Sine}}

\bigskip

$\sin (A+B) = \sin A \cos B + \cos A \sin B$

\vskip 0.5 cm

$\sin (2 \theta) = \sin (\theta + \theta) = $

\vskip 0.5 cm

$= $

\vskip 0.5 cm

\vfill

{\bf {Double Angle Formula for Cosine}}

\bigskip

$\cos (A+B) = \cos A \cos B - \sin A \sin B$

\vskip 0.8 cm

$\cos (2 \theta) = \cos (\theta + \theta) = $

\vskip 0.8 cm

$= $

\vskip 0.8 cm

Find two more formulas for $\cos (2\theta)$ by using the Alternate Versions of the Pythagorean Theorem

\vskip 0.8 cm

$\cos (2\theta) = $

\vskip 0.8 cm

$\cos (2\theta) = $

\newpage

{\bf {Double Angle Formula for Tangent}}

\bigskip

$\tan (A+B) = \displaystyle{\frac{\tan A + \tan B}{1-\tan A \tan B}}$

\vskip 0.5 cm

$\tan (2 \theta) = \tan (\theta + \theta) = $

\vskip 1 cm

$= $

\vskip 0.5 cm

\vfill

{\bf {Half Angle Formula for Sine}}

\bigskip

Begin with $\cos(2A) = 1 - 2 \sin^2(A)$. Let $2A = \theta$. Rewrite the previous equation in terms of theta and solve for $\sin(\frac{\theta}{2})$.

\vskip 0.5 cm

$\cos(\theta) = $

\vfill

{\bf {Half Angle Formula for Cosine}}

\bigskip

Begin with $\cos(2A) = 2 \cos^2(A) - 1$. Let $2A = \theta$. Rewrite the previous equation in terms of theta and solve for $\cos(\frac{\theta}{2})$.

\vskip 0.5 cm

$\cos(\theta) = $

\vfill

{\bf {Half Angle Formula for Tangent}}

\bigskip

$\displaystyle{\tan\left(\frac{\theta}{2}\right) = \frac{\sin(\frac{\theta}{2})}{\cos(\frac{\theta}{2})} = }$

\vfill

\end{document}


%% \begin{center}\begin{tabular}{|c|c|}
%%   \hline
%%   cell1 & cell2 \\
%%   cell3 & cell4 \\
%%   \hline
%%   \end{tabular}\end{center}

%%   \begin{center}\begin{tabular}{|c|c|}
%%     \hline
%%     cell1 & cell2 \\
%%     cell3 & cell4 \\
%%     \hline
%%     \end{tabular}\end{center}


\documentclass[numbers]{ximera}

%My github password is ximera1
%My GeoGebra password is Ximera1


%\usepackage[fleqn]{amsmath}

\title{Double Angle Formulas}
\author{Mike Weimerskirch}
\begin{document}
\begin{abstract}
  In this activity, you will discover formulas for $\sin (2\theta)$, $\cos (2\theta)$, $\sin \left(\frac{\theta}{2}\right)$ and $\cos \left(\frac{\theta}{2}\right)$, using your knowledge of the Angle Sum Formulas.
\end{abstract}

\maketitle


Recall the Pythagorean Theorem:

\[\sin^2 \theta + \cos^2 \theta = 1\]

We can use the Pythagorean theorem to derive expressions for $\sin^2\theta$ and for $\cos^2\theta$.

\begin{problem}
Subtract $\cos^2 \theta$ from both sides to get a formula for $\sin^2 \theta$:

$\sin^2 \theta =\answer{1-\cos^2 \theta} $
\end{problem}

\begin{problem}
Subtract $\sin^2 \theta$ from both sides to get a formula for $\cos^2 \theta$:

$\cos^2\theta = \answer{1-\sin^2\theta}$
\end{problem}

We can derive double-angle formulas using the angle-sum formulas from a previous activity.

\begin{problem}
Rewrite the following values as a multiple of 2:
\begin{itemize}
	\item $4=2\times\answer{2}$
	\item $8=2\times\answer{4}$
	\item $70=2\times\answer{35}$
	\item $\theta = 2 \times \answer{\frac{\theta}{2}}$
	\item $2\theta = 2\times\answer{\theta}$
\end{itemize}
\end{problem}

\begin{problem}
Rewrite the following multiples of $2$ as the sum of a value with itself:
\begin{itemize}
	\item $4=2+\answer{2}$
	\item $8=\answer{4}+\answer{4}$
	\item $70=\answer{35}+\answer{35}$
	\item $\theta = \answer{\frac{\theta}{2}}+\answer{\frac{\theta}{2}}$
	\item $2\theta=\answer{\theta}+\answer{\theta}$
\end{itemize}
\end{problem}

\begin{question}
Describe the relationship between doubling a number and adding a number to itself.
\begin{freeResponse}
\end{freeResponse}
\end{question}


We can now combine all of these ideas to derive the double-angle formulas for sine and cosine.

Recall the angle-sum formula for sine:
\[\sin \left(A+B\right) = \sin A \cos B + \cos A \sin B\]
We seek an expression for $\sin\left(2\theta\right)$.

\begin{problem}
Use a substitution you made above to rewrite $\sin\left(2\theta\right)$ as the sine of the sum of two angles:

$\sin\left(2\theta\right)=\sin\left(\answer{\theta}+\answer{\theta}\right)$
\end{problem}

\begin{problem}
What substitutions can you make for $A$ and $B$ in order to apply the angle-sum formula for sine to $\sin\left(2\theta\right)$?
\begin{itemize}
	\item $A=\answer{\theta}$
	\item $B=\answer{\theta}$
\end{itemize}
\end{problem}

\begin{problem}
Rewrite the angle-sum formula using your substitutions from Problem 7:
\[\sin\left(\theta+\theta\right)=\sin\answer{\theta}\cos\answer{\theta}+\cos\answer{\theta}\sin\answer{\theta}\]
\end{problem}

\begin{question}
Hence,
\[\sin\left(2\theta\right)=\answer{2\sin\theta\cos\theta}\]
\begin{hint}
$\sin\theta\cos\theta=\cos\theta\sin\theta$
\end{hint}
%% Will this be interpreted as the same expression as 2\sin\theta\cos\theta?
\end{question}

You will now discover the double-angle formula for cosine.

Recall the angle-sum formula for cosine:
\[\cos \left(A+B\right) = \cos A \cos B - \sin A \sin B\]
We seek an expression for $\cos\left(2\theta\right)$.

\begin{problem}
Rewrite $\cos\left(2\theta\right)$ as the cosine of the sum of two angles:

$\cos\left(2\theta\right)=\cos\left(\answer{\theta}+\answer{\theta}\right)$
\end{problem}

\begin{problem}
What substitutions can you make for $A$ and $B$ in order to apply the angle-sum formula for cosine to $\cos\left(2\theta\right)$?
\begin{itemize}
	\item $A=\answer{\theta}$
	\item $B=\answer{\theta}$
\end{itemize}
\end{problem}

\begin{problem}
Plug your values from above into the angle-sum formula for cosine:
\[\cos\left(\theta+\theta\right) = \cos\answer{\theta}\cos\answer{\theta}-\sin\answer{\theta}\sin\answer{\theta}\]
\end{problem}

\begin{question}
Hence,
\[\cos\left(2\theta\right)=\answer{\cos^2\theta-\sin^2\theta}\]
\begin{hint}
$\cos\theta\cos\theta=\cos^2\theta$
\end{hint}
\end{question}

You can find two more formulas for $\cos \left(2\theta\right)$ by using the Alternate Versions of the Pythagorean Theorem from the start of this activity.

\begin{problem}
Write an expression for $\cos\left(2\theta\right)$ that does not involve $\cos\theta$:

$\cos\left(2\theta\right) = \answer{1-2\sin^2\theta}$
\begin{hint}
$\cos^2\theta = 1 - \sin^2\theta$
\end{hint}
\end{problem}

\begin{problem}
Write an expression for $\cos\left(2\theta\right)$ that does not involve $\sin\theta$:

$\cos\left(2\theta\right) = \answer{2\cos^2\theta-1}$
\begin{hint}
$\sin^2\theta=1-\cos^2\theta$
\end{hint}
\end{problem}

%%%%%%%%%%%%%$$$$$$$
% Double Angle Formula for Tangent
%%%%%%%%%%%%%$$$$$$$

Using these same techniques, you can discover a double-angle formula for tangent.
Recall from a previous activity that
\[\tan \left(A+B\right) = \displaystyle{\frac{\tan A + \tan B}{1-\tan A \tan B}}\]

\begin{problem}
Substitute $A=\theta$ and $B=\theta$ to derive an expression for $\tan\left(2\theta\right)=\tan\left(\theta+\theta\right)$.
\[\tan\left(\theta+\theta\right)=\frac{\tan\answer{\theta}+\tan\answer{\theta}}{1-\tan\answer{\theta}\tan\answer{\theta}}\]
\end{problem}

\begin{question}
Hence,
\[\tan\left(2\theta\right)=\answer{\frac{2\tan\theta}{1-\tan^2\theta}}\]
\begin{hint}
$\tan\theta+\tan\theta=2\tan\theta$
\end{hint}

\begin{hint}
$\tan\theta\tan\theta=\tan^2\theta$
\end{hint}

\begin{hint}
Be mindful of parentheses when writing fractions! $a/1-b$ and $\frac{a}{1-b}$ are {\bf not} the same expression.
\end{hint}
\end{question}

If you can take the sine, cosine, and tangent of two-times-an angle, it might be natural to ask if you take the sine, cosine, and tangent of half of an angle.

\begin{problem}
Rewrite the following values as half of another value.
\begin{itemize}
	\item $2=\frac{\answer{4}}{2}$
	\item $4=\frac{\answer{8}}{2}$
	\item $35=\frac{\answer{70}}{2}$
	\item $\frac{\theta}{2} = \frac{\answer{\theta}}{2}$
	\item $\theta=\frac{\answer{2\theta}}{2}$

\end{itemize}
\end{problem}

First you will derive the half-angle formula for sine.

\begin{problem}
Begin with $\cos\left(2A\right) = 1 - 2 \sin^2\left(A\right)$. Let $2A = \theta$. 
Rewrite the previous equation in terms of $\theta$:

$\cos\left(\theta\right) = \answer{1-2\sin^2\left(\frac{\theta}{2}\right)}$
\begin{hint}
Your answer should include $\sin\left(\frac{\theta}{2}\right)$.
\end{hint}
\end{problem}

\begin{question}
Solve for $\sin\left(\frac{\theta}{2}\right)$ in the previous expression:
\[\sin\left(\frac{\theta}{2}\right)=\answer{\pm\sqrt{\frac{1-\cos\theta}{2}}}\]
\begin{hint} Subtract $1$ from both sides. \end{hint}
\begin{hint} Divide by $-2$ and distribute the $-$ sign. \end{hint}
\begin{hint} Take the square root of both sides. Don't forget the $\pm$ sign! \end{hint}
\end{question}

Next you will derive a half-angle formula for cosine.

\begin{problem}
Begin with $\cos\left(2A\right) = 2 \cos^2\left(A\right) - 1$. Let $2A = \theta$.
Rewrite $\cos\left(2A\right)$ in terms of $\theta$:

$\cos\left(\theta\right)=\answer{2\cos^2\left(\frac{\theta}{2}\right)-1}$
\begin{hint}
Your answer should include $\cos\left(\frac{\theta}{2}\right)$.
\end{hint}
\end{problem}

\begin{question}
Solve for $\cos\left(\frac{\theta}{2}\right)$ in the previous expression:
\[\cos\left(\frac{\theta}{2}\right)=\answer{\pm \sqrt{\frac{\cos\theta+1}{2}}}\]
\begin{hint} Subtract $1$ from both sides. \end{hint}
\begin{hint} Divide by $2$. \end{hint}
\begin{hint} Take the square root of both sides. Don't forget the $\pm$ sign! \end{hint}
\end{question}

Lastly, you will discover the half-angle formula for tangent.

\begin{question}
Recalling that $\tan\theta=\frac{\sin\theta}{\cos\theta}$, derive the half-angle formula for tangent:

\[\tan\left(\frac{\theta}{2}\right) = \frac{\sin\left(\frac{\theta}{2}\right)}{\cos\left(\frac{\theta}{2}\right)} = \answer{\pm\sqrt{\frac{1-\cos\theta}{1+\cos\theta}}}\]
\begin{hint} Use your expressions for $\sin\left(\frac{\theta}{2}\right)$ and $\cos\left(\frac{\theta}{2}\right)$ as the numerator and denominator of a fraction. \end{hint}
\begin{hint} Square roots distribute over division, e.g., $\frac{\sqrt{a}}{\sqrt{b}}=\sqrt{\frac{a}{b}}$ \end{hint}
\begin{hint} Dividing by a fraction is the same as multiplying by the reciprocal of that fraction. \end{hint}
\end{question}
\vfill

\end{document}



\documentclass[number]{ximera}

%My github password is ximera1
%My GeoGebra password is Ximera1
%\usepackage{tikz}

\renewcommand{\theenumi}{\arabic{enumi}.}


% set font encoding for PDFLaTeX, XeLaTeX, or LuaTeX
\usepackage{ifxetex,ifluatex}
\newif\ifxetexorluatex
\ifxetex
  \xetexorluatextrue
\else
  \ifluatex
    \xetexorluatextrue
  \else
    \xetexorluatexfalse
  \fi
\fi

\ifxetexorluatex
  \usepackage{fontspec}
\else
  \usepackage[T1]{fontenc}
  \usepackage[utf8]{inputenc}
  \usepackage{lmodern}
\fi

\usepackage{hyperref}
\usepackage{tikz} %% Adding this here so it's not manually added elsewhere

\title{Solving Triangles}

\begin{document}

\begin{abstract}
  In this activity, you will discover the issue involved in using the Law of Sines and Law of Cosines to solve triangles.
\end{abstract}

\maketitle

In this activity, you will be given three parts of a triangle (side lengths and/or angle measures) and will be asked to place points to produce a triangle that has these three parts. The questions are intentionally open ended, there may be many different ways to produce triangles with the given parts. When working with your group members, you should try to produce original answers. For example, if your group members put their red points high, have your red point low.

\bigskip

Two triangles are the same if they are congruent. That is, two triangles are the same if you can move one triangle to line up with another by sliding, rotating and/or flipping. Another way to describe this is if one student lists the side lengths and angle measures in order around the triangle, they should match up, in the same order with a congruent version. The congruent version may start in a different location, and may go clockwise instead of counterclockwise.

%In the interactive window below, the red and blue points can be moved by using the mouse. Simply click-and-drag a point to a new location. You can also move a point by moving the slider bars using the same click-and-drag technique.

In the interactive window below, you may change the length of the blue side $b$, the red side $c$, and the green angle $A$. Simply click-and-drag the slider bar to a new value. 

 $b$ controls the $x$-coordinate of the red dot (and also the length of the blue side), $c$ controls the length of the red side, and $A$ controls the angle at the green dot. You may also move the slider bars by slowly clicking on the slider bar. For example, clicking on the $b$ control bar to the right of the circle indicator will increase the $x$-coordinate of the red dot. You will use the same interactive window for all nine parts of this exploration.

%\HCode{<iframe scrolling="no" src="https://www.geogebra.org/material/iframe/id/u9G9V7HU/width/791/height/469/border/888888" width="791px" height="469px" style="border:0px;"> </iframe>}

%To embed a GeoGebra widget in a ximera document:
%Get the ID of the widget on GeoGebraTube. This is the last part of the widget?s URL. For example, in the URL https://www.geogebra.org/m/DGFA47ym the ID is DGFA47ym. It doesn?t matter who made the widget.
%Embed the widget using the syntax \geogebra{ID}{width}{height}, where ID is the widget ID and width and height are the dimensions (in pixels) you want the embedded widget to have. For example, try \geogebra{DGFA47ym}{640}{480}.
%This command takes several options which change how the widget is embedded. These are boolean flags, with default value False. To set any of them to True just include a list of flag names as an optional argument, like \geogebra[rc,sdz]{}{}{}
%rc : Enable right click
%sdz : Enable pan and zoom
%smb : Show menu
%stb : Show toolbar
%ld : Enable dragging labels
%sri : Show reset icon

\geogebra[sdz]{tkj5MtNe}{800}{480}


Exercise 1

{\bf {Instructions:}} Adjust the sliders until the three sides of the triangle have lengths 5, 7 and 8. 

\begin{question}
How many different triangles can be produced?
\begin{multipleChoice}
\choice{There are no such triangles.}
\choice[correct]{There is one unique triangle.}
\choice{There are two distinct triangles.}
\choice{There are many such triangles.}
\end{multipleChoice}

\begin{question}
(Note: Round angles to the nearest $0.1^\circ$)

What is the degree measure of the largest angle? $\answer[tolerance=0.5]{81.8}^\circ$

What is the degree measure of the smallest angle? $\answer[tolerance=0.5]{38.2}^\circ$

What is the degree measure of the mid-sized angle? $\answer[tolerance=0.5]{59.9}^\circ$
\end{question}
\end{question}

\bigskip

Exercise 2

Adjust the sliders until the three sides of the triangle have lengths 3, 4 and 8. 

\begin{question}
How many different triangles can be produced?
\begin{multipleChoice}
\choice[correct]{There are no such triangles.}
\choice{There is one unique triangle.}
\choice{There are two distinct triangles.}
\choice{There are many such triangles.}
\end{multipleChoice}

\end{question}

\bigskip

Exercise 3

Adjust the sliders until the three angles of the triangle have measures $52^\circ$, $85^\circ$ and $43^\circ$. 

\begin{question}
How many different triangles can be produced?
\begin{multipleChoice}
\choice{There are no such triangles.}
\choice{There is one unique triangle.}
\choice{There are two distinct triangles.}
\choice[correct]{There are many such triangles.}
\end{multipleChoice}

\end{question}

\bigskip

Exercise 4

Adjust the sliders until one angle of the triangle has measure $73^\circ$ and the sides on either side of the $73^\circ$ angle have lengths 4 and 6.

\begin{question}
How many different triangles can be produced?
\begin{multipleChoice}
\choice{There are no such triangles.}
\choice[correct]{There is one unique triangle.}
\choice{There are two distinct triangles.}
\choice{There are many such triangles.}
\end{multipleChoice}
\begin{question}
(Note: Round angles to the nearest $0.1^\circ$. Round lengths to the nearest 0.01)

What is the degree measure of the larger remaining angle? $\answer[tolerance=0.5]{68.62}^\circ$

What is the degree measure of the smaller remaining angle? $\answer[tolerance=0.5]{38.38}^\circ$

What is the length of the remaining side? $\answer[tolerance=0.05]{6.16}$
\end{question}
\end{question}

Exercise 5

Adjust the sliders until two angles of the triangle have measures $84^\circ$ and $63^\circ$, and the side between the two given angles has length 5.

\begin{question}
How many different triangles can be produced?
\begin{multipleChoice}
\choice{There are no such triangles.}
\choice[correct]{There is one unique triangle.}
\choice{There are two distinct triangles.}
\choice{There are many such triangles.}
\end{multipleChoice}
\begin{question}
(Note: Round angles to the nearest $0.1^\circ$. Round lengths to the nearest 0.01)

What is the degree measure of the remaining angle? $\answer[tolerance=0.5]{33}^\circ$

What is the length of the shorter of the two remaining sides? $\answer[tolerance=0.5]{8.18}$

What is the length of the longer of the two remaining sides? $\answer[tolerance=0.05]{9.13}$
\end{question}
\end{question}

Exercise 6

Adjust the sliders until two angles of the triangle have measures $42^\circ$ and $77^\circ$, and the side across from the $77^\circ$ angle has length 9.

\begin{question}
How many different triangles can be produced?
\begin{multipleChoice}
\choice{There are no such triangles.}
\choice[correct]{There is one unique triangle.}
\choice{There are two distinct triangles.}
\choice{There are many such triangles.}
\end{multipleChoice}
\begin{question}
(Note: Round angles to the nearest $0.1^\circ$. Round lengths to the nearest 0.01)

What is the degree measure of the remaining angle? $\answer[tolerance=0.5]{61}^\circ$

What is the length of the shorter of the two remaining sides? $\answer[tolerance=0.5]{6.18}$

What is the length of the longer of the two remaining sides? $\answer[tolerance=0.05]{8.08}$
\end{question}
\end{question}

Exercise 7

Adjust the sliders until one angle of the triangle has measure $52^\circ$, a side adjacent to the $52^\circ$ angle has length 8 and the side across from the $52^\circ$ angle has length 6.4.

\begin{question}
How many different triangles can be produced?
\begin{multipleChoice}
\choice{There are no such triangles.}
\choice{There is one unique triangle.}
\choice[correct]{There are two distinct triangles.}
\choice{There are many such triangles.}
\end{multipleChoice}
\begin{question}
(Note: Round angles to the nearest $0.1^\circ$. Round lengths to the nearest 0.01)

In the triangle with an obtuse angle (an angle larger than $90^\circ$): 

what is the degree measure of the largest angle? $\answer[tolerance=0.5]{99.93}^\circ$

what is the degree measure of the smallest angle? $\answer[tolerance=0.5]{28.06}^\circ$

what is the length of the remaining side? $\answer[tolerance=0.5]{3.84}$

In the triangle without an obtuse angle:

what is the degree measure of the larger remaining angle? $\answer[tolerance=0.5]{80.5}^\circ$

what is the degree measure of the smaller remaining angle? $\answer[tolerance=0.5]{47.5}^\circ$

what is the length of the remaining side? $\answer[tolerance=0.5]{5.98}$

\end{question}
\end{question}

Exercise 8

Adjust the sliders until one angle of the triangle has measure $52^\circ$, a side adjacent to the $52^\circ$ angle has length 8 and the side across from the $52^\circ$ angle has length 9.

\begin{question}
How many different triangles can be produced?
\begin{multipleChoice}
\choice{There are no such triangles.}
\choice[correct]{There is one unique triangle.}
\choice{There are two distinct triangles.}
\choice{There are many such triangles.}
\end{multipleChoice}
\begin{question}
(Note: Round angles to the nearest $0.1^\circ$. Round lengths to the nearest 0.01)

What is the degree measure of the larger remaining angle? $\answer[tolerance=0.5]{83.54}^\circ$

What is the degree measure of the smaller remaining angle? $\answer[tolerance=0.5]{44.46}^\circ$

What is the length of the remaining side? $\answer[tolerance=0.5]{11.35}$

\end{question}
\end{question}

Exercise 9

Adjust the sliders until one angle of the triangle has measure $52^\circ$, a side adjacent to the $52^\circ$ angle has length 8 and the side across from the $52^\circ$ angle has length 5.

\begin{question}
How many different triangles can be produced?
\begin{multipleChoice}
\choice[correct]{There are no such triangles.}
\choice{There is one unique triangle.}
\choice{There are two distinct triangles.}
\choice{There are many such triangles.}
\end{multipleChoice}

\end{question}

\end{document}


